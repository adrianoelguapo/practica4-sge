\documentclass[12pt]{article}
\usepackage[spanish]{babel}
\usepackage[letterpaper,top=2cm,bottom=2cm,left=3cm,right=3cm,marginparwidth=1.75cm]{geometry}
\usepackage{tabularx}
\usepackage{fancyvrb}
\usepackage{graphicx}
\usepackage{setspace}
\usepackage{ragged2e}
\usepackage[T1]{fontenc}
\renewcommand*\familydefault{\sfdefault}
\usepackage{librefranklin}

\setcounter{tocdepth}{2}
\setlength{\parindent}{0pt}
\onehalfspacing

\begin{document}

\begin{titlepage}

    \centering
    \vspace*{1 cm}
    \Huge
    \textbf{Práctica 4}

    \vspace{0.5 cm}
    \Large
    Sistemas de Gestión Empresarial

    \vspace{5.5 cm}
    \textbf{Adrián Condines Celada}

    \vspace{0.8 cm}    
    \normalsize
    Aula Estudio\\

    \vspace{0.8 cm}
    2º Ciclo Superior - Desarrollo de Aplicaciones Multiplataforma\\

    \vspace{0.8 cm}
    Curso 2025 - 2026

\end{titlepage}

\justify
\tableofcontents
\newpage

\section{INTRODUCCIÓN Y OBJETIVOS}

\justify
El objetivo de esta práctica es la modificación y creación de módulos personalizados en Odoo 

\section{PASO PREVIO PARA ACTIVIDADES 1 y 2}

\justify
Para las actividades 1 y 2, previamente se clonará el repositorio que contiene los módulos de ejemplo que serán necesarios para la realización de dichas actividades. Para ello se abrirá una terminal en el directorio en el que se desea clonar el repositorio y se ejecutará el siguiente comando: \newline

\centering
\texttt{git clone https://github.com/sergarb1/OdooModulosEjemplos}

\begin{figure}[h]

    \centering
    \includegraphics[width=15cm]{images/1.png}

\end{figure}

\justify
\section{ACTIVIDAD 1}

\subsection{Introducción}

\justify
El objetivo de esta actividad es modificar el ejemplo más básico de la \textbf{Lista de Tareas} \texttt{(Directorio EJ02-ListaTareas del repositorio clonado previamente)} para que las tareas se muestren en formato \textbf{Kanban}. También se debe crear una nueva vista para visualizar las tareas en formato \textbf{Calendario} según la fecha definida.

\newpage

\subsection{Modelos}

\justify
Partiendo del directorio mencionado previamente, el primer paso será modificar el archivo \texttt{models/models.py} para añadir el nuevo campo \textbf{Fecha} al modelo de la lista de tareas.

\justify
Para definir el modelo se crea una nueva clase de Python que hereda de la clase \textbf{models.Model} y también se han definido los siguientes elementos:

\begin{itemize}

    \item  \textbf{\_name}: Define como como Odoo guardará el modelo en la base de datos.
    \item  \textbf{\_description}: Descripción del modelo.
    \item  \textbf{\_rec\_name}: Campo a mostrar por defecto en las vistas y menús.
    \item  \textbf{tarea}: Nombre de la tarea \texttt{(String)}.
    \item  \textbf{prioridad}: Nivel de prioridad de la tarea \texttt{(Integer)}.
    \item  \textbf{urgente}: Atributo computado que define si la tarea es urgente o no \texttt{(Booleano)}.
    \item  \textbf{realizada}: Define si la tarea ha sido realizada o no \texttt{(Booleano)}.
    \item  \textbf{\_value\_urgente}: Método que se encarga de calcular el valor del atributo 'urgente' en función del valor del atributo 'prioridad'.

\end{itemize}

\begin{figure}[h]

    \centering
    \includegraphics[width=15cm]{images/2.png}

\end{figure}

\newpage

\subsection{Vistas}

\justify
Una vez modificado el modelo, hay que actualizar las vistas del módulo para que reflejen los cambios realizado en el modelo y también para añadir los nuevos tipos de vistas requeridos para la realización de la actividad. Para ello se modificará el archivo \texttt{views/views.xml}.

\justify
El primer paso sera definir la acción principal de módulo que será abrir el modelo \texttt{lista\_tareas.lista} y muestra sus vistas. Se deben de definir los siguientes campos:

\begin{itemize}

    \item  \textbf{name}: Título visible de la ventana.
    \item  \textbf{res\_model}: Modelo al que se refiere la acción.
    \item  \textbf{view\_mode}: Tipo de vistas que se mostrarán: list \texttt{(tabla)}, form \texttt{(formulario)}, kanban \texttt{(kanban)}, calendar \texttt{(calendario)}.

\end{itemize}

\begin{figure}[h]

    \centering
    \includegraphics[width=15cm]{images/3.png}

\end{figure}

\justify
A continuación se definirán los menús del módulo que aparecerán en la parte superior de la interfaz de Odoo y para ello se les debe asignar un atributo \textbf{id} que será el identificador único de cada menú, un atributo \textbf{name} que será el nombre visble del menú y un atributo \textbf{parent} que será el menú padre del menú. 

\newpage

\justify
Para ello, se definirán los siguientes \texttt{menuitem}:

\begin{itemize}

    \item  \textbf{lista\_tareas\_menu\_root}: Menú raíz del módulo.
    \item  \textbf{lista\_tareas\_menu\_1}: Submenú dentro del menú raíz que puede agrupar más opciones si se llegarán a necesitar.
    \item  \textbf{lista\_tareas\_menu\_1\_list}: Opción del menú que ejecuta la acción principal del módulo definida en el paso anterior (se le añade el atributo \textbf{action}, que hace referencia a la acción del módulo).
    
\end{itemize}

\begin{figure}[h]

    \centering
    \includegraphics[width=15cm]{images/4.png}

\end{figure}

\justify
Una vez definidas las acciones y los menús, se procederá a crear y explicar todas las vistas del modelo. La primera de ellas será la vista de tipo \textbf{list} que mostrará la lista de tareas en una tabla y para ello se utiliza la etiqueta \textbf{record} con los atributos \textbf{id} (identificador único de la vista), \textbf{model} (modelo de la vista).

\justify
Dentro de la etiqueta record se definen los siguientes campos:

\begin{itemize}

    \item  \textbf{name}: Nombre interno de la vista.
    \item  \textbf{model}: Modelo utilizado enla vista.
    \item  \textbf{arch}: Define la estructura XML que tendrá la tarea.

\end{itemize}

\newpage

\justify
Dentro del campo con \texttt{name:} \textbf{arch} se define el tipo de vista y su orden (se ordenará en función del valor del atributo \textbf{prioridad} y en orden descendente), que en este caso será \textbf{list} y también se definen los campos que se mostrarán en la vista:

\begin{itemize}

    \item  \textbf{tarea}: Nombre de la tarea \texttt{(String)}.
    \item  \textbf{prioridad}: Nivel de prioridad de la tarea \texttt{(Integer)}.
    \item  \textbf{fecha}: Fecha límite de la tarea \texttt{(Date)}.
    \item  \textbf{urgente}: Atributo computado que define si la tarea es urgente o no \texttt{(Booleano)}.
    \item  \textbf{realizada}: Define si la tarea ha sido realizada o no \texttt{(Booleano)}.

\end{itemize}

\justify
Así quedaría la vista de tipo \textbf{list} \texttt{(views/views.xml)}:

\begin{figure}[h]

    \centering
    \includegraphics[width=15cm]{images/5.png}

\end{figure}

\newpage

\justify
La siguiente vista a explicar es la vista de de tipo \textbf{form} que se mostrará cuando el usuario cree o edite una tarea. Para ello, al igual que con la vista anterior, se debe utilizar la etiqueta \textbf{record} con los atributos \textbf{id} (identificador único de la vista), \textbf{model} (modelo de la vista).

\justify
Dentro de la etiqueta record se definen los siguientes campos:

\begin{itemize}

    \item  \textbf{name}: Nombre interno de la vista.
    \item  \textbf{model}: Modelo utilizado en la vista.
    \item  \textbf{arch}: Define la estructura XML que tendrá la tarea.

\end{itemize}

\justify
Dentro del campo con \texttt{name:} \textbf{arch} se define el tipo de vista y su orden que en este caso será \textbf{form} con el atributo \textbf{name} para darle un título al formulario. En el interior de la etiqueta que define el tipo de vista se definirá una etiqueta \textbf{sheet}, que actúa como el contenedor principal del formulario.

\justify
Dentro de la etiqueta \textbf{sheet}, están definidos dos grupos de campos y en ellos se definen los siguientes atributos:

\begin{itemize}

    \item  \textbf{tarea}: Nombre de la tarea \texttt{(String)}.
    \item  \textbf{fecha}: Fecha límite de la tarea \texttt{(Date)}.
    \item  \textbf{realizada}: Define si la tarea ha sido realizada o no \texttt{(Booleano)}.
    \item  \textbf{prioridad}: Nivel de prioridad de la tarea \texttt{(Integer)}.
    \item  \textbf{urgente}: Atributo computado que define si la tarea es urgente o no \texttt{(Booleano)}.

\end{itemize}

\newpage

\justify
Así quedaría la vista de tipo \textbf{form} \texttt{(views/views.xml)}:

\begin{figure}[h]

    \centering
    \includegraphics[width=15cm]{images/6.png}

\end{figure}

\newpage

\justify
La siguiente vista a explicar es la vista de tipo \textbf{calendar} y para ello como en todas las vistas se define con la etiqueta \textbf{record} con los atributos \textbf{id} (identificador único de la vista), \textbf{model} (modelo de la vista).

\justify
Dentro de la etiqueta record se definen los siguientes campos:

\begin{itemize}

    \item  \textbf{name}: Nombre interno de la vista.
    \item  \textbf{model}: Modelo utilizado en la vista.
    \item  \textbf{arch}: Define la estructura XML que tendrá la tarea.

\end{itemize}

\justify
Dentro del campo con \texttt{name:} \textbf{arch} se define el tipo de vista y su orden que en este caso será \textbf{calendar} con los siguientes atributos:

\begin{itemize}

    \item  \textbf{string}: Título de la vista.
    \item  \textbf{date\_start}: Fecha de inicio de la tarea \texttt{(Date)}.
    \item  \textbf{color}: Adignación de colores a las tareas según su prioridad.
    \item  \textbf{mode}: Define el modo de la vista que en este caso será dividida por meses.

\end{itemize}

\justify
En la vista de calendario se mostrarán los siguientes datos de la tarea:

\begin{itemize}

    \item  \textbf{tarea}: Nombre de la tarea \texttt{(String)}.
    \item  \textbf{prioridad}: Nivel de prioridad de la tarea \texttt{(Integer)}.

\end{itemize}

\newpage

\justify
Así quedaría la vista de tipo \textbf{calendar} \texttt{(views/views.xml)}:

\begin{figure}[h]

    \centering
    \includegraphics[width=15cm]{images/7.png}

\end{figure}

\justify
La siguiente vista a explicar es la vista de tipo \textbf{kanban} y para ello como en todas las vistas se define con la etiqueta \textbf{record} con los atributos \textbf{id} (identificador único de la vista), \textbf{model} (modelo de la vista).

\justify
Dentro de la etiqueta record se definen los siguientes campos:

\begin{itemize}

    \item  \textbf{name}: Nombre interno de la vista.
    \item  \textbf{model}: Modelo utilizado en la vista.
    \item  \textbf{arch}: Define la estructura XML que tendrá la tarea.

\end{itemize}

\newpage

\justify
Dentro del campo con \texttt{name:} \textbf{arch} se define el tipo de vista, que en este caso será \textbf{kanban} con los siguientes atributos:

\begin{itemize}

    \item  \textbf{tarea}: Título de la vista.
    \item  \textbf{fecha}: Fecha límite de la tarea \texttt{(Date)}.
    \item  \textbf{prioridad}: Nivel de prioridad de la tarea \texttt{(Integer)}.

\end{itemize}

\begin{figure}[h]

    \centering
    \includegraphics[width=15cm]{images/8.png}

\end{figure}

\newpage

\justify
Posteriormente se diseña una plantilla para mostrar cada una de las tareas en la vista, quedando así la plantilla para la vista de tipo \textbf{kanban}:

\begin{figure}[h]

    \centering
    \includegraphics[width=8.5cm]{images/9.png}

\end{figure}

\newpage

\justify
La última vista que queda por explicar es la vista de tipo \textbf{search} y para ello como en todas las vistas se define con la etiqueta \textbf{record} con los atributos \textbf{id} (identificador único de la vista), \textbf{model} (modelo de la vista).

\justify
Dentro de la etiqueta record se definen los siguientes campos:

\begin{itemize}

    \item  \textbf{name}: Nombre interno de la vista.
    \item  \textbf{model}: Modelo utilizado en la vista.
    \item  \textbf{arch}: Define la estructura XML que tendrá la tarea.

\end{itemize}

\justify
Dentro del campo con \texttt{name:} \textbf{arch} se define la estructura XML que tendrá la tarea. En este caso será del tipo \textbf{search} con el atributo \textbf{string} que define el título de la vista. 

\justify
En esta vista el único campo que interesa es el nombre de la tarea ya que los demás campos no son relevantes para una búsqueda de texto y también se añaden filtros para mostrar las tareas que son urgentes y las que ya están realizadas, quedando así la vista de tipo \textbf{search}:

\begin{figure}[h]

    \centering
    \includegraphics[width=15cm]{images/10.png}

\end{figure}

\newpage

\subsection{Funcionamiento}

\justify
Ya terminada la explicación del modelo y las vistas del módulo, se pondrá a prueba su funcionamiento y para ello el primer paso es pasar el directorio del módulo del equipo local al contenedor de Odoo, concretamente a la carpeta mapeada para almacenar los módulos extra añadidos por el usuario y para ello se utilizará el siguiente comando: \newline

\centering
\texttt{docker cp <ruta\_módulo> <contenedor\_odoo>:/mnt/extra-addons/<directorio>}

\begin{figure}[h]

    \centering
    \includegraphics[width=15cm]{images/11.png}

\end{figure}

\justify
Una vez que se haya copiado el módulo al contenedor de Odoo, se debe activar el \textbf{Modo Desarrollador} y actualizar la lista de aplicaciones (explicado en la práctica anterior). Posteriormente se debe buscar el módulo en la lista de aplicaciones e instalarlo pulsando sobre el botón lila de \textbf{Activar} (se instalará el primer módulo de la lista ya que el segundo es el de la práctica anterior).

\begin{figure}[h]

    \centering
    \includegraphics[width=15cm]{images/12.png}

\end{figure}

\justify
Una vez instalado el módulo, se accederá a su interfaz mediante el menú de Odoo y se crearán nuevas tareas para comprobar su funcionamiento. La vista de creación de tarea (\textbf{form}) se ve de la siguiente manera:

\begin{figure}[h]

    \centering
    \includegraphics[width=15cm]{images/13.png}

\end{figure}

\newpage

\justify
Como se puede comprobar, las tareas se crean correctamente y así se muestran en la vista de tipo \textbf{list}:

\begin{figure}[h]

    \centering
    \includegraphics[width=15cm]{images/14.png}

\end{figure}

\justify
Así se mostrarían las tareas en la vista de tipo \textbf{kanban}:

\begin{figure}[h]

    \centering
    \includegraphics[width=15cm]{images/15.png}

\end{figure}

\justify
Y por último, así se mostrarían las tareas en la vista de tipo \textbf{calendar}:

\begin{figure}[h]

    \centering
    \includegraphics[width=15cm]{images/16.png}

\end{figure}

\newpage

\section{ACTIVIDAD 2}

\subsection{Introducción}

\justify
El objetivo de esta actividad es amplicar el módulo de ejemplo de la \textbf{Biblioteca} \texttt{Directorio EJ03-ComicsSimple del repositorio clonado previamente} para que incluya la posibilidad de incluir \textbf{Socios}. También se debe implementar un sistema de gestión de \textbf{Préstamos de Cómics} para que los socios de la biblioteca puedan llevarse cómics prestados.

\subsection{Modelos}

\subsubsection{Modelo Socio}

\justify
El primer paso para crear el modelo es definir el nombre de la clase que definirá al modelo y que heredará de la clase \textbf{models.Model}. Una vez definida la clase, se definen los siguientes elementos:

\begin{itemize}

    \item \textbf{\_name}: Define como Odoo guardará el modelo en la base de datos.
    \item \textbf{\_description}: Descripción del modelo.
    \item \textbf{\_rec\_name}: Campo a mostrar por defecto en las vistas y menús.
    \item \textbf{nombre}: Nombre del socio \texttt{(String)}.
    \item \textbf{apellido}: Apellido del socio \texttt{(String)}.
    \item \textbf{identificador}: Identificador del socio \texttt{(String)}.
    
\end{itemize}

\justify
Así quedaría el modelo \textbf{Socio} \texttt{(models/biblioteca\_socio.py)}:

\begin{figure}[h]

    \centering
    \includegraphics[width=12cm]{images/71.png}

\end{figure}

\subsubsection{Modelo Ejemplar}

\justify
El primer paso para crear el modelo es definir el nombre de la clase que definirá al modelo y que heredará de la clase \textbf{models.Model}. Una vez definida la clase, se definen los siguientes elementos:

\begin{itemize}

    \item \textbf{\_name}: Define como Odoo guardará el modelo en la base de datos.
    \item \textbf{\_description}: Descripción del modelo.
    \item \textbf{\_rec\_name}: Campo a mostrar por defecto en las vistas y menús.
    \item \textbf{comic\_id}: Relación con el modelo de Comic \texttt{(Many2one)}.
    \item \textbf{socio\_id}: Relación con el modelo de Socio \texttt{(Many2one)}.
    \item \textbf{fecha\_inicio}: Fecha de inicio del préstamo \texttt{(Date)}.
    \item \textbf{fecha\_fin}: Fecha de fin del préstamo \texttt{(Date)}.
    
\end{itemize}

\justify
Además, se añade una restricción (\texttt{\_check\_fechas}) para comprobar que las fechas son coherentes: la fecha de inicio no puede ser posterior al día actual y la fecha de devolución no puede ser anterior al día actual.

\justify
Así quedaría el modelo \textbf{Ejemplar} \texttt{(models/biblioteca\_ejemplar.py)}:

\begin{figure}[h]

    \centering
    \includegraphics[width=12cm]{images/72.png}

\end{figure}

\subsubsection{Modelo Cómic}

\justify
Se ha modificado el modelo de \textbf{biblioteca.comic} para añadir nuevos campos y funcionalidades. Se definen los siguientes elementos:

\begin{itemize}

    \item \textbf{estado}: Estado del cómic (borrador, disponible, perdido) \texttt{(Selection)}.
    \item \textbf{descripcion}: Descripción del cómic \texttt{(Html)}.
    \item \textbf{portada}: Imagen de la portada \texttt{(Binary)}.
    \item \textbf{fecha\_publicacion}: Fecha de publicación \texttt{(Date)}.
    \item \textbf{precio}: Precio del cómic \texttt{(Float)}.
    \item \textbf{paginas}: Número de páginas \texttt{(Integer)}.
    \item \textbf{valoracion\_lector}: Valoración media de los lectores \texttt{(Float)}.
    \item \textbf{autor\_ids}: Autores del cómic \texttt{(Many2many)}.
    
\end{itemize}

\justify
También se incluyen restricciones SQL para asegurar que el título sea único y que el número de páginas sea positivo, así como una restricción de python para validar la fecha de publicación.

\newpage

\justify
Así quedaría el modelo \textbf{Biblioteca Comic} \texttt{(models/biblioteca\_comic.py)}:

\begin{figure}[h]

    \centering
    \includegraphics[width=12cm]{images/73.png}

\end{figure}

\newpage











\subsection{Vistas}

\subsubsection{Vistas de Cómic}

\justify
El primer paso sera definir la acción principal que será abrir el modelo \texttt{biblioteca.comic} y muestra sus vistas. Se deben de definir los siguientes campos:

\begin{itemize}

    \item \textbf{name}: Título visible de la ventana.
    \item \textbf{res\_model}: Modelo al que se refiere la acción.
    \item \textbf{view\_mode}: Tipo de vistas que se mostrarán: list \texttt{(tabla)}, form \texttt{(formulario)}.

\end{itemize}

\begin{figure}[h]

    \centering
    \includegraphics[width=15cm]{images/82.png}

\end{figure}

\justify
Una vez definida la acción principal, se procederá a crear y explicar todas las vistas del modelo. La primera de ellas será la vista de tipo \textbf{list}, que mostrará los cómics en una tabla. Para ello se definen los siguientes elementos:

\begin{itemize}

    \item \textbf{name}: Título visible de la ventana.
    \item \textbf{model}: Modelo al que pertenece esta vista.
    \item \textbf{arch}: Estructura XML de la vista.

\end{itemize}

\justify
Dentro del campo con \texttt{name: arch} se define el tipo de vista que en este caso será de tipo \textbf{list} y también se definen los campos que se mostrarán en la vista:

\begin{itemize}

    \item \textbf{nombre}: Nombre del cómic \texttt{(String)}.
    \item \textbf{fecha\_publicacion}: Fecha de publicación del cómic \texttt{(Date)}.
    \item \textbf{estado}: Estado del cómic \texttt{(Selection)}.

\end{itemize}

\newpage

\justify
Así quedaría la vista de tipo \textbf{list} \texttt{(views/biblioteca\_comic.xml)}:

\begin{figure}[h]

    \centering
    \includegraphics[width=15cm]{images/83.png}

\end{figure}

\justify
La siguiente vista a explicar de este modelo será la vista de tipo \textbf{form}, utilizada para crear y editar cómics. Se definen los siguientes elementos:

\begin{itemize}

    \item \textbf{name}: Título visible de la ventana.
    \item \textbf{model}: Modelo al que pertenece esta vista.
    \item \textbf{arch}: Estructura XML de la vista.

\end{itemize}

\newpage

\justify
Dentro del campo con \texttt{name: arch} se define el tipo de vista que en este caso será de tipo \textbf{form}. Se utliza una etiqueta \textbf{sheet} como contenedor principal del formulario. Además, se incluye una cabecera (\textbf{header}) con un botón para archivar cómics. 

\justify
También se definen los campos que se mostrarán en la vista para poder ser rellenados:

\begin{itemize}

    \item \textbf{nombre}: Nombre del cómic \texttt{(String)}.
    \item \textbf{autor\_ids}: Autores del cómic \texttt{(Many2many)}.
    \item \textbf{estado}: Estado del cómic \texttt{(Selection)}.
    \item \textbf{paginas}: Número de páginas del cómic \texttt{(Integer)}.
    \item \textbf{activo}: Campo de solo lectura que indica si el cómic está activo \texttt{(Boolean)}.
    \item \textbf{precio}: Precio del cómic \texttt{(Float)}.
    \item \textbf{fecha\_publicacion}: Fecha de publicación del cómic \texttt{(Date)}.
    \item \textbf{portada}: Imagen de la portada del cómic \texttt{(Binary)}.
    \item \textbf{valoracion\_lector}: Valoración media de los lectores \texttt{(Float)}.
    \item \textbf{descripcion}: Descripción del cómic \texttt{(Html)}.

\end{itemize}

\newpage

\justify
Así quedaría la vista de tipo \textbf{form} \texttt{(views/biblioteca\_comic.xml)}:

\begin{figure}[h]

    \centering
    \includegraphics[width=15cm]{images/84.png}

\end{figure}

\justify
La última vista a explicar de este modelo será la vista de tipo \textbf{search}, utilizada para buscar cómics. Se definen los siguientes elementos:

\begin{itemize}

    \item \textbf{name}: Título visible de la ventana.
    \item \textbf{model}: Modelo al que pertenece esta vista.
    \item \textbf{arch}: Estructura XML de la vista.

\end{itemize}

\newpage

\justify
Dentro del campo con \texttt{name: arch} se define el tipo de vista que en este caso será de tipo \textbf{search}. Se definen los campos de búsqueda y un filtro predefinido para mostrar los cómics sin autor:

\begin{itemize}

    \item \textbf{nombre}: Campo de búsqueda por nombre del cómic \texttt{(String)}.
    \item \textbf{autor\_ids}: Campo de búsqueda por autores \texttt{(Many2many)}.
    \item \textbf{Filtro "Sin autor"}: Filtro que muestra solo los cómics que no tienen autor asignado.

\end{itemize}

\justify
Así quedaría la vista de tipo \textbf{search} \texttt{(views/biblioteca\_comic.xml)}:

\begin{figure}[h]

    \centering
    \includegraphics[width=15cm]{images/85.png}

\end{figure}

\newpage

\subsubsection{Vistas de Socio}

\justify
El primer paso sera definir la acción principal que será abrir el modelo \texttt{biblioteca.socio} y muestra sus vistas. Se deben de definir los siguientes campos:

\begin{itemize}

    \item \textbf{name}: Título visible de la ventana.
    \item \textbf{res\_model}: Modelo al que se refiere la acción.
    \item \textbf{view\_mode}: Tipo de vistas que se mostrarán: list \texttt{(tabla)}, form \texttt{(formulario)}.

\end{itemize}

\begin{figure}[h]

    \centering
    \includegraphics[width=15cm]{images/86.png}

\end{figure}

\justify
Una vez definida la acción principal, se procederá a crear y explicar todas las vistas del modelo. La primera de ellas será la vista de tipo \textbf{list}, que mostrará los socios en una tabla. Para ello se definen los siguientes elementos:

\begin{itemize}

    \item \textbf{name}: Título visible de la ventana.
    \item \textbf{model}: Modelo al que pertenece esta vista.
    \item \textbf{arch}: Estructura XML de la vista.

\end{itemize}

\newpage

\justify
Dentro del campo con \texttt{name: arch} se define el tipo de vista que en este caso será de tipo \textbf{list} y también se definen los campos que se mostrarán en la vista:

\begin{itemize}

    \item \textbf{nombre}: Nombre del socio \texttt{(String)}.
    \item \textbf{apellido}: Apellido del socio \texttt{(String)}.
    \item \textbf{identificador}: Identificador del socio \texttt{(String)}.

\end{itemize}

\justify
Así quedaría la vista de tipo \textbf{list} \texttt{(views/biblioteca\_socio.xml)}:

\begin{figure}[h]

    \centering
    \includegraphics[width=15cm]{images/87.png}

\end{figure}

\justify
La última vista a explicar de este modelo será la vista de tipo \textbf{form}, utilizada para crear y editar socios. Se definen los siguientes elementos:

\begin{itemize}

    \item \textbf{name}: Título visible de la ventana.
    \item \textbf{model}: Modelo al que pertenece esta vista.
    \item \textbf{arch}: Estructura XML de la vista.

\end{itemize}

\newpage

\justify
Dentro del campo con \texttt{name: arch} se define el tipo de vista que en este caso será de tipo \textbf{form}. Se utliza una etiqueta \textbf{group} como contenedor de los campos y también se definen los campos que se mostrarán en la vista para poder ser rellenados:

\begin{itemize}

    \item \textbf{nombre}: Nombre del socio \texttt{(String)}.
    \item \textbf{apellido}: Apellido del socio \texttt{(String)}.
    \item \textbf{identificador}: Identificador del socio \texttt{(String)}.

\end{itemize}

\justify
Así quedaría la vista de tipo \textbf{form} \texttt{(views/biblioteca\_socio.xml)}:

\begin{figure}[h]

    \centering
    \includegraphics[width=15cm]{images/88.png}

\end{figure}

\newpage

\subsubsection{Vistas de Préstamo}

\justify
El primer paso sera definir la acción principal que será abrir el modelo \texttt{biblioteca.ejemplar} y muestra sus vistas. Se deben de definir los siguientes campos:

\begin{itemize}

    \item \textbf{name}: Título visible de la ventana.
    \item \textbf{res\_model}: Modelo al que se refiere la acción.
    \item \textbf{view\_mode}: Tipo de vistas que se mostrarán: list \texttt{(tabla)}, form \texttt{(formulario)}.

\end{itemize}

\begin{figure}[h]

    \centering
    \includegraphics[width=15cm]{images/89.png}

\end{figure}

\justify
Una vez definida la acción principal, se procederá a crear y explicar todas las vistas del modelo. La primera de ellas será la vista de tipo \textbf{list}, que mostrará los préstamos en una tabla. Para ello se definen los siguientes elementos:

\begin{itemize}

    \item \textbf{name}: Título visible de la ventana.
    \item \textbf{model}: Modelo al que pertenece esta vista.
    \item \textbf{arch}: Estructura XML de la vista.

\end{itemize}

\newpage

\justify
Dentro del campo con \texttt{name: arch} se define el tipo de vista que en este caso será de tipo \textbf{list} y también se definen los campos que se mostrarán en la vista:

\begin{itemize}

    \item \textbf{comic\_id}: Cómic prestado \texttt{(Many2one)}.
    \item \textbf{socio\_id}: Socio que toma el préstamo \texttt{(Many2one)}.
    \item \textbf{fecha\_inicio}: Fecha de inicio del préstamo \texttt{(Date)}.
    \item \textbf{fecha\_fin}: Fecha de fin del préstamo \texttt{(Date)}.

\end{itemize}

\justify
Así quedaría la vista de tipo \textbf{list} \texttt{(views/biblioteca\_ejemplar.xml)}:

\begin{figure}[h]

    \centering
    \includegraphics[width=15cm]{images/90.png}

\end{figure}

\justify
La última vista a explicar de este modelo será la vista de tipo \textbf{form}, utilizada para crear y editar préstamos. Se definen los siguientes elementos:

\begin{itemize}

    \item \textbf{name}: Título visible de la ventana.
    \item \textbf{model}: Modelo al que pertenece esta vista.
    \item \textbf{arch}: Estructura XML de la vista.

\end{itemize}

\justify
Dentro del campo con \texttt{name: arch} se define el tipo de vista que en este caso será de tipo \textbf{form}. Se utliza una etiqueta \textbf{group} como contenedor de los campos y también se definen los campos que se mostrarán en la vista para poder ser rellenados:

\begin{itemize}

    \item \textbf{comic\_id}: Cómic prestado \texttt{(Many2one)}.
    \item \textbf{socio\_id}: Socio que toma el préstamo \texttt{(Many2one)}.
    \item \textbf{fecha\_inicio}: Fecha de inicio del préstamo \texttt{(Date)}.
    \item \textbf{fecha\_fin}: Fecha de fin del préstamo \texttt{(Date)}.

\end{itemize}

\justify
Así quedaría la vista de tipo \textbf{form} \texttt{(views/biblioteca\_ejemplar.xml)}:

\begin{figure}[h]

    \centering
    \includegraphics[width=15cm]{images/91.png}

\end{figure}











\newpage

\subsection{Funcionamiento}

\justify
Ya terminada la explicación de los modelos y las vistas del módulo, se pondrá a prueba su funcionamiento y para ello el primer paso es pasar el directorio del módulo del equipo local al contenedor de Odoo, concretamente a la carpeta mapeada para almacenar los módulos extra añadidos por el usuario y para ello se utilizará el siguiente comando: \newline

\centering
\texttt{docker cp <ruta\_módulo> <contenedor\_odoo>:/mnt/extra-addons/<directorio>}

\begin{figure}[h]

    \centering
    \includegraphics[width=15cm]{images/74.png}

\end{figure}

\justify
Una vez que se haya copiado el módulo al contenedor de Odoo, se debe activar el \textbf{Modo Desarrollador} y actualizar la lista de aplicaciones (explicado en la práctica anterior). Posteriormente se debe buscar el módulo en la lista de aplicaciones e instalarlo pulsando sobre el botón lila de \textbf{Activar}.

\begin{figure}[h]

    \centering
    \includegraphics[width=15cm]{images/75.png}

\end{figure}

\justify
Una vez instalado el módulo, se accederá a su interfaz mediante el menú de Odoo y se creará un nuevo cómic. La vista de creación y edición de cómic (\textbf{form}) se ve de la siguiente manera:

\begin{figure}[h]

    \centering
    \includegraphics[width=15cm]{images/76.png}

\end{figure}

\newpage

\justify
Al crear el cómic, se muestra de la siguiente manera en la vista de tipo \textbf{list}:

\begin{figure}[h]

    \centering
    \includegraphics[width=10cm]{images/77.png}

\end{figure}

\justify
Una vez creado un cómic, es hora de crear un nuevo socio de la biblioteca. La vista de creación y edición de socio \textbf{(form)} se ve de la siguiente manera:

\begin{figure}[h]

    \centering
    \includegraphics[width=10cm]{images/78.png}

\end{figure}

\justify
Al crear el nuevo socio, se muestra de la siguiente manera en la vista de tipo \textbf{list}:

\begin{figure}[h]

    \centering
    \includegraphics[width=10cm]{images/79.png}

\end{figure}

\newpage

\justify
Ahora que ya hay definidos un ejemplar de un cómic y un socio, es hora de simular un préstamo de ese mismo cómic a ese mismo socio. La vista de creación y edición de préstamo \textbf{(form)} se ve de la siguiente manera:

\begin{figure}[h]

    \centering
    \includegraphics[width=10cm]{images/80.png}

\end{figure}

\justify
Al crear el nuevo préstamo, se muestra de la siguiente manera en la vista de tipo \textbf{list}:

\begin{figure}[h]

    \centering
    \includegraphics[width=10cm]{images/81.png}

\end{figure}

\newpage

\justify
\section{ACTIVIDAD 3}

\subsection{Introducción}

\justify
El objetivo de esta actividad es crear un módulo para simular un \textbf{Sistema de Consultas} de un hospital en el que intervienen \textbf{Médicos} y \textbf{Pacientes}. 

\subsection{Base del Módulo}

\justify
Para ello el primer paso es crear la base del módulo utilizando \textbf{Odoo Scaffold} y para ello se utilizará el siguiente comando: \newline

\centering
\texttt{docker exec -it <contenedor\_odoo> odoo scaffold <nombre\_directorio> <ruta\_addons>}

\begin{figure}[h]

    \centering
    \includegraphics[width=15cm]{images/17.png}

\end{figure}

\justify
Una vez creada la base de módulo, se debe pasar el directorio del módulo del contenedor de Odoo al equipo local para poder trabajar con él y para ello se utilizará el siguiente comando: \newline

\centering
\texttt{docker cp <contenedor\_odoo>:<directorio\_módulo> <ruta\_directorio\_local>}

\begin{figure}[h]

    \centering
    \includegraphics[width=15cm]{images/18.png}

\end{figure}

\justify
\subsection{Modelos}

\subsubsection{Modelo Paciente}

\justify
El primer paso para crear el modelo es definir el nombre de la clase que definirá al modelo y que heredará de la clase \textbf{models.Model}. Una vez definida la clase, se definen los siguientes elementos:

\begin{itemize}

    \item \textbf{\_name}: Define como Odoo guardará el modelo en la base de datos.
    \item \textbf{\_description}: Descripción del modelo.
    \item \textbf{\_rec\_name}: Campo a mostrar por defecto en las vistas y menús.
    \item \textbf{symptoms}: Atributo que define los síntomas del paciente \texttt{(String)}.
    \item \textbf{consulta\_ids}: Atributo que define las consultas del paciente \texttt{(Many2one)}.

\end{itemize}

\newpage

\justify
Así quedaría el modelo \textbf{Paciente} \texttt{(models/paciente.py)}:

\begin{figure}[h]

    \centering
    \includegraphics[width=16cm]{images/21.png}

\end{figure}

\subsubsection{Modelo Médico}

\justify
El primer paso para crear el modelo es definir el nombre de la clase que definirá al modelo y que heredará de la clase \textbf{models.Model}. Una vez definida la clase, se definen los siguientes elementos:

\begin{itemize}

    \item \textbf{\_name}: Define como Odoo guardará el modelo en la base de datos.
    \item \textbf{\_description}: Descripción del modelo.
    \item \textbf{name}: Campo a mostrar por defecto en las vistas y menús.
    \item \textbf{numero\_colegiado}: Atributo que define el número de colegiado del médico \texttt{(String)}.
    \item \textbf{consulta\_ids}: Atributo que define las consultas del médico \texttt{(Many2one)}.

\end{itemize}

\newpage

\justify
Así quedaría el modelo \textbf{Médico} \texttt{(models/medico.py)}:

\begin{figure}[h]

    \centering
    \includegraphics[width=16cm]{images/22.png}

\end{figure}

\subsubsection{Modelo Consulta}

\justify
El primer paso para crear el modelo es definir el nombre de la clase que definirá al modelo y que heredará de la clase \textbf{models.Model}. Una vez definida la clase, se definen los siguientes elementos:

\begin{itemize}

    \item \textbf{\_name}: Define como Odoo guardará el modelo en la base de datos.
    \item \textbf{\_description}: Descripción del modelo.
    \item \textbf{paciente\_id}: Atributo que define el paciente de la consulta \texttt{(Many2one)}.
    \item \textbf{medico\_id}: Atributo que define el médico de la consulta \texttt{(Many2one)}.
    \item \textbf{diagnostico}: Atributo que define el diagnostico de la consulta \texttt{(Text)}.
    \item \textbf{fecha\_hora}: Atributo que define la fecha y hora de la consulta \texttt{(Datetime)}.

\end{itemize}

\newpage

\justify
Así quedaría el modelo \textbf{Consulta} \texttt{(models/consulta.py)}:

\begin{figure}[h]

    \centering
    \includegraphics[width=16cm]{images/23.png}

\end{figure}

\justify
\subsection{Vistas}

\subsubsection{Vistas de Paciente}

\justify
El primer paso sera definir la acción principal que será abrir el modelo \texttt{hospital.paciente} y muestra sus vistas. Se deben de definir los siguientes campos:

\begin{itemize}

    \item \textbf{name}: Título visible de la ventana.
    \item \textbf{res\_model}: Modelo al que se refiere la acción.
    \item \textbf{view\_mode}: Tipo de vistas que se mostrarán: list \texttt{(tabla)}, form \texttt{(formulario)}.

\end{itemize}

\begin{figure}[h]

    \centering
    \includegraphics[width=16cm]{images/24.png}

\end{figure}

\newpage

\justify
Una vez definida la acción principal, se procederá a crear y explicar todas las vistas del modelo. La primera de ellas será la vista de tipo \textbf{list}, que mostrará los pacientes en una tabla. Para ello se definen los siguientes elementos:

\begin{itemize}

    \item \textbf{name}: Título visible de la ventana.
    \item \textbf{model}: Modelo al que pertenece esta vista.
    \item \textbf{arch}: Estructura XML de la vista.

\end{itemize}

\justify
Dentro del campo con \texttt{name: arch} se define el tipo de vista que en este caso será de tipo \textbf{list} y también se definen los campos que se mostrarán en la vista:

\begin{itemize}

    \item \textbf{name}: Nombre del paciente \texttt{(String)}.
    \item \textbf{symptoms}: Síntomas del paciente \texttt{(Text)}.

\end{itemize}

\justify
Así quedaría la vista de tipo \textbf{list} \texttt{(views/paciente\_list.xml)}:

\begin{figure}[h]

    \centering
    \includegraphics[width=16cm]{images/25.png}

\end{figure}

\newpage

\justify
La última vista a explicar de este modelo será la vista de tipo \textbf{form}, utilizada para crear y editar pacientes. Se definen los siguientes elementos:

\begin{itemize}

    \item \textbf{name}: Título visible de la ventana.
    \item \textbf{model}: Modelo al que pertenece esta vista.
    \item \textbf{arch}: Estructura XML de la vista.

\end{itemize}

\justify
Dentro del campo con \texttt{name: arch} se define el tipo de vista que en este caso será de tipo \textbf{form}. Se utliza una etiqueta \textbf{sheet} como contenedor principal del formulario y también se definen los campos que se mostrarán en la vista para poder ser rellenados:

\begin{itemize}

    \item \textbf{name}: Nombre del paciente \texttt{(String)}.
    \item \textbf{symptoms}: Síntomas del paciente \texttt{(Text)}.

\end{itemize}

\justify
Así quedaría la vista de tipo \textbf{form} \texttt{(views/paciente\_form.xml)}:

\begin{figure}[h]

    \centering
    \includegraphics[width=16cm]{images/26.png}

\end{figure}

\subsubsection{Vistas de Médico}

\justify
El primer paso sera definir la acción principal que será abrir el modelo \texttt{hospital.medico} y muestra sus vistas. Se deben de definir los siguientes campos:

\begin{itemize}

    \item \textbf{name}: Título visible de la ventana.
    \item \textbf{res\_model}: Modelo al que se refiere la acción.
    \item \textbf{view\_mode}: Tipo de vistas que se mostrarán: list \texttt{(tabla)}, form \texttt{(formulario)}.

\end{itemize}

\begin{figure}[h]

    \centering
    \includegraphics[width=13cm]{images/27.png}

\end{figure}

\justify
Una vez definida la acción principal, se procederá a crear y explicar todas las vistas del modelo. La primera de ellas será la vista de tipo \textbf{list}, que mostrará los médicos en una tabla. Para ello se definen los siguientes elementos:

\begin{itemize}

    \item \textbf{name}: Título visible de la ventana.
    \item \textbf{model}: Modelo al que pertenece esta vista.
    \item \textbf{arch}: Estructura XML de la vista.

\end{itemize}

\justify
Dentro del campo con \texttt{name: arch} se define el tipo de vista que en este caso será de tipo \textbf{list} y también se definen los campos que se mostrarán en la vista:

\begin{itemize}

    \item \textbf{name}: Nombre del médico \texttt{(String)}.
    \item \textbf{numero\_colegiado}: Número de colegiado \texttt{(String)}.

\end{itemize}

\newpage

\justify
Así quedaría la vista de tipo \textbf{list} \texttt{(views/medico\_views.xml)}:

\begin{figure}[h]

    \centering
    \includegraphics[width=16cm]{images/28.png}

\end{figure}

\justify
La última vista a explicar de este modelo será la vista de tipo \textbf{form}, utilizada para crear y editar médicos. Se definen los siguientes elementos:

\begin{itemize}

    \item \textbf{name}: Título visible de la ventana.
    \item \textbf{model}: Modelo al que pertenece esta vista.
    \item \textbf{arch}: Estructura XML de la vista.

\end{itemize}

\justify
Dentro del campo con \texttt{name: arch} se define el tipo de vista que en este caso será de tipo \textbf{form}. Se utliza una etiqueta \textbf{sheet} como contenedor principal del formulario y también se definen los campos que se mostrarán en la vista para poder ser rellenados:

\begin{itemize}

    \item \textbf{name}: Nombre del médico \texttt{(String)}.
    \item \textbf{numero\_colegiado}: Número de colegiado \texttt{(String)}.

\end{itemize}

\newpage

\justify
Así quedaría la vista de tipo \textbf{form} \texttt{(views/medico\_views.xml)}:

\begin{figure}[h]

    \centering
    \includegraphics[width=16cm]{images/29.png}

\end{figure}

\newpage

\subsubsection{Vistas de Consulta}

\justify
El primer paso sera definir la acción principal que será abrir el modelo \texttt{hospital.consulta} y muestra sus vistas. Se deben de definir los siguientes campos:

\begin{itemize}

    \item \textbf{name}: Título visible de la ventana.
    \item \textbf{res\_model}: Modelo al que se refiere la acción.
    \item \textbf{view\_mode}: Tipo de vistas que se mostrarán: list \texttt{(tabla)}, form \texttt{(formulario)}.

\end{itemize}

\begin{figure}[h]

    \centering
    \includegraphics[width=13cm]{images/30.png}

\end{figure}

\justify
Una vez definida la acción principal, se procederá a crear y explicar todas las vistas del modelo. La primera de ellas será la vista de tipo \textbf{list}, que mostrará las consultas en una tabla. Para ello se definen los siguientes elementos:

\begin{itemize}

    \item \textbf{name}: Título visible de la ventana.
    \item \textbf{model}: Modelo al que pertenece esta vista.
    \item \textbf{arch}: Estructura XML de la vista.

\end{itemize}

\justify
Dentro del campo con \texttt{name: arch} se define el tipo de vista que en este caso será de tipo \textbf{list} y también se definen los campos que se mostrarán en la vista:

\begin{itemize}

    \item \textbf{paciente\_id}: Paciente a atender \texttt{(String)}.
    \item \textbf{medico\_id}: Médico que atiende al paciente \texttt{(String)}.
    \item \textbf{fecha\_hora}: Fecha y hora de la consulta \texttt{(Datetime)}.
    \item \textbf{diagnostico}: Diagnóstico \texttt{(Text)}.

\end{itemize}

\newpage

\justify
Así quedaría la vista de tipo \textbf{list} \texttt{(views/consulta\_views.xml)}:

\begin{figure}[h]

    \centering
    \includegraphics[width=16cm]{images/31.png}

\end{figure}

\subsubsection{Vistas de Menús}

\justify
A continuación se definirán los menús del módulo que aparecerán en la parte superior de la interfaz de Odoo y para ello se les debe asignar un atributo \textbf{id} que será el identificador único de cada menú, un atributo \textbf{name} que será el nombre visble del menú y un atributo \textbf{parent} que será el menú padre del menú. 

\newpage

\justify
Para ello, se definirán los siguientes \texttt{menuitem}:

\begin{itemize}

    \item \textbf{menu\_hospital}: Menú principal que aparece en la barra de navegación.
    \item \textbf{menu\_hospital\_pacientes}: Submenú del modelo paciente que aparece en el menú principal.
    \item \textbf{menu\_hospital\_medicos}: Submenú del modelo médico que aparece en el menú principal.
    \item \textbf{menu\_hospital\_consultas}: Submenú del modelo consulta que aparece en el menú principal.

\end{itemize}

\justify
Así quedarían los menús \texttt{(views/menu\_views.xml)}:

\begin{figure}[h]

    \centering
    \includegraphics[width=15cm]{images/32.png}

\end{figure}

\justify
\subsection{Otros archivos}

\justify
Se debe modificar el archivo \texttt{models/\_\_init\_\_.py} para importar los modelos que se han creado.

\begin{figure}[h]

    \centering
    \includegraphics[width=7cm]{images/33.png}

\end{figure}

\newpage

\justify
Se debe modificar el archivo \texttt{\_\_manifest\_\_.py} para importar las vistas que se han creado.

\begin{figure}[h]

    \centering
    \includegraphics[width=15cm]{images/34.png}

\end{figure}



























\subsection{Funcionamiento}

\justify
Ya terminada la explicación de los modelos y las vistas del módulo, se pondrá a prueba su funcionamiento y para ello el primer paso es pasar el directorio del módulo del equipo local al contenedor de Odoo, concretamente a la carpeta mapeada para almacenar los módulos extra añadidos por el usuario y para ello se utilizará el siguiente comando: \newline

\centering
\texttt{docker cp <ruta\_módulo> <contenedor\_odoo>:/mnt/extra-addons/<directorio>}

\begin{figure}[h]

    \centering
    \includegraphics[width=15cm]{images/53.png}

\end{figure}

\newpage

\justify
Una vez que se haya copiado el módulo al contenedor de Odoo, se debe activar el \textbf{Modo Desarrollador} y actualizar la lista de aplicaciones (explicado en la práctica anterior). Posteriormente se debe buscar el módulo en la lista de aplicaciones e instalarlo pulsando sobre el botón lila de \textbf{Activar}.

\begin{figure}[h]

    \centering
    \includegraphics[width=15cm]{images/54.png}

\end{figure}

\justify
Una vez instalado el módulo, se accederá a su interfaz mediante el menú de Odoo y se crearán nuevos pacientes. La vista de creación y edición de paciente (\textbf{form}) se ve de la siguiente manera:

\begin{figure}[h]

    \centering
    \includegraphics[width=15cm]{images/55.png}

\end{figure}

\justify
Una vez creado el paciente, se podrá ver su información en la vista de tipo \textbf{list} que se ve de la siguiente manera:

\begin{figure}[h]

    \centering
    \includegraphics[width=15cm]{images/56.png}

\end{figure}

\newpage

\justify
Ya creados un par de pacientes, se procederá a crear un nuevo médico para atenderlos. La vista de creación y edición de médico (\textbf{form}) se ve de la siguiente manera:

\begin{figure}[h]

    \centering
    \includegraphics[width=15cm]{images/57.png}

\end{figure}

\justify
Una vez creado el médico, se podrá ver su información en la vista de tipo \textbf{list} que se ve de la siguiente manera:

\begin{figure}[h]

    \centering
    \includegraphics[width=15cm]{images/58.png}

\end{figure}

\newpage

\justify
Ahora que se han creado los pacientes y los médicos, se procederá a crear la consulta para cada paciente. La vista de creación y edición de consulta (\textbf{form}) se ve de la siguiente manera:

\begin{figure}[h]

    \centering
    \includegraphics[width=15cm]{images/59.png}

\end{figure}

\justify
Una vez creado la consulta, se podrá ver su información en la vista de tipo \textbf{list} que se ve de la siguiente manera:

\begin{figure}[h]

    \centering
    \includegraphics[width=15cm]{images/60.png}

\end{figure}

\newpage

\justify
\section{ACTIVIDAD 4}

\subsection{Introducción}

\justify
El objetivo de esta actividad es desarrollar un módulo para simular la gestión de \textbf{Ciclos Formativos} en un instituto en el que se tendrán en cuenta los mismos Ciclos Formativos, los \textbf{Módulos} del ciclo, los \textbf{Profesores} que imparten dichos ciclos y los \textbf{Alumnos} matriculados.

\subsection{Base del Módulo}

\justify
Para ello el primer paso es crear la base del módulo utilizando \textbf{Odoo Scaffold} y para ello se utilizará el siguiente comando: \newline

\centering
\texttt{docker exec -it <contenedor\_odoo> odoo scaffold <nombre\_directorio> <ruta\_addons>}

\begin{figure}[h]

    \centering
    \includegraphics[width=15cm]{images/19.png}

\end{figure}

\justify
Una vez creada la base de módulo, se debe pasar el directorio del módulo del contenedor de Odoo al equipo local para poder trabajar con él y para ello se utilizará el siguiente comando: \newline

\centering
\texttt{docker cp <contenedor\_odoo>:<directorio\_módulo> <ruta\_directorio\_local>}

\begin{figure}[h]

    \centering
    \includegraphics[width=15cm]{images/20.png}

\end{figure}

\newpage

\justify
\subsection{Modelos}

\subsubsection{Modelo Alumno}

\justify
El primer paso para crear el modelo es definir el nombre de la clase que definirá al modelo y que heredará de la clase \textbf{models.Model}. Una vez definida la clase, se definen los siguientes elementos:

\begin{itemize}

    \item \textbf{\_name}: Define como Odoo guardará el modelo en la base de datos.
    \item \textbf{\_description}: Descripción del modelo.
    \item \textbf{name}: Atributo que define el nombre completo del alumno \texttt{(String)}.
    \item \textbf{dni}: Atributo que define el DNI del alumno \texttt{(String)}.
    \item \textbf{modulo\_ids}: Atributo que define los módulos que cursa el alumno \texttt{(Many2many)}.

\end{itemize}

\justify
Así quedaría el modelo \textbf{Alumno} \texttt{(models/alumno.py)}:

\begin{figure}[h]

    \centering
    \includegraphics[width=12cm]{images/35.png}

\end{figure}

\newpage

\subsubsection{Modelo Profesor}

\justify
El primer paso para crear el modelo es definir el nombre de la clase que definirá al modelo y que heredará de la clase \textbf{models.Model}. Una vez definida la clase, se definen los siguientes elementos:

\begin{itemize}

    \item \textbf{\_name}: Define como Odoo guardará el modelo en la base de datos.
    \item \textbf{\_description}: Descripción del modelo.
    \item \textbf{name}: Atributo que define el nombre completo del profesor \texttt{(String)}.
    \item \textbf{dni}: Atributo que define el DNI del profesor \texttt{(String)}.
    \item \textbf{modulo\_ids}: Atributo que define los módulos que imparte el profesor \texttt{(One2many)}.

\end{itemize}

\justify
Así quedaría el modelo \textbf{Profesor} \texttt{(models/profesor.py)}:

\begin{figure}[h]

    \centering
    \includegraphics[width=12cm]{images/36.png}

\end{figure}

\newpage

\subsubsection{Modelo Ciclo Formativo}

\justify
El primer paso para crear el modelo es definir el nombre de la clase que definirá al modelo y que heredará de la clase \textbf{models.Model}. Una vez definida la clase, se definen los siguientes elementos:

\begin{itemize}

    \item \textbf{\_name}: Define como Odoo guardará el modelo en la base de datos.
    \item \textbf{\_description}: Descripción del modelo.
    \item \textbf{name}: Atributo que define el nombre completo del ciclo formativo \texttt{(String)}.
    \item \textbf{codigo}: Atributo que define el código del ciclo formativo \texttt{(String)}.
    \item \textbf{modulo\_ids}: Atributo que define los módulos que forman el ciclo formativo \texttt{(One2many)}.

\end{itemize}

\justify
Así quedaría el modelo \textbf{Ciclo Formativo} \texttt{(models/ciclo\_formativo.py)}:

\begin{figure}[h]

    \centering
    \includegraphics[width=12cm]{images/37.png}

\end{figure}

\newpage

\subsubsection{Modelo Módulo}

\justify
El primer paso para crear el modelo es definir el nombre de la clase que definirá al modelo y que heredará de la clase \textbf{models.Model}. Una vez definida la clase, se definen los siguientes elementos:

\begin{itemize}

    \item \textbf{\_name}: Define como Odoo guardará el modelo en la base de datos.
    \item \textbf{\_description}: Descripción del modelo.
    \item \textbf{name}: Atributo que define el nombre completo del módulo \texttt{(String)}.
    \item \textbf{codigo}: Atributo que define el código del módulo \texttt{(String)}.
    \item \textbf{ciclo\_ids}: Atributo que define los módulos que imparte el profesor \texttt{(Many2One)}.
    \item \textbf{profesor\_ids}: Atributo que define el profesor que imparte el módulo \texttt{(Many2One)}.
    \item \textbf{alumno\_ids}: Atributo que define los alumnos que cursan el módulo \texttt{(Many2many)}.

\end{itemize}

\justify
Así quedaría el modelo \textbf{Módulo} \texttt{(models/modulo.py)}:

\begin{figure}[h]

    \centering
    \includegraphics[width=12cm]{images/38.png}

\end{figure}

\subsection{Vistas}

\subsubsection{Vistas de Alumno}

\justify
El primer paso sera definir la acción principal que será abrir el modelo \texttt{ciclos.alumno} y muestra sus vistas. Se deben de definir los siguientes campos:

\begin{itemize}

    \item \textbf{name}: Título visible de la ventana.
    \item \textbf{res\_model}: Modelo al que se refiere la acción.
    \item \textbf{view\_mode}: Tipo de vistas que se mostrarán: list \texttt{(tabla)}, form \texttt{(formulario)}.

\end{itemize}

\begin{figure}[h]

    \centering
    \includegraphics[width=13cm]{images/39.png}

\end{figure}

\justify
Una vez definida la acción principal, se procederá a crear y explicar todas las vistas del modelo. La primera de ellas será la vista de tipo \textbf{list}, que mostrará los alumnos en una tabla. Para ello se definen los siguientes elementos:

\begin{itemize}

    \item \textbf{name}: Título visible de la ventana.
    \item \textbf{model}: Modelo al que pertenece esta vista.
    \item \textbf{arch}: Estructura XML de la vista.

\end{itemize}

\justify
Dentro del campo con \texttt{name: arch} se define el tipo de vista que en este caso será de tipo \textbf{list} y también se definen los campos que se mostrarán en la vista:

\begin{itemize}

    \item \textbf{name}: Nombre completo del alumno \texttt{(String)}.
    \item \textbf{dni}: DNI del alumno \texttt{(String)}.

\end{itemize}

\newpage

\justify
Así quedaría la vista \textbf{list} \texttt{(views/alumno\_views.xml)}:

\begin{figure}[h]

    \centering
    \includegraphics[width=13cm]{images/43.png}

\end{figure}

\justify
La última vista a explicar de este modelo será la vista de tipo \textbf{form}, utilizada para crear y editar alumnos. Se definen los siguientes elementos:

\begin{itemize}

    \item \textbf{name}: Título visible de la ventana.
    \item \textbf{model}: Modelo al que pertenece esta vista.
    \item \textbf{arch}: Estructura XML de la vista.

\end{itemize}

\justify
Dentro del campo con \texttt{name: arch} se define el tipo de vista que en este caso será de tipo \textbf{form}. Se utliza una etiqueta \textbf{sheet} como contenedor principal del formulario y también se definen los campos que se mostrarán en la vista para poder ser rellenados:

\begin{itemize}

    \item \textbf{name}: Nombre completo del alumno \texttt{(String)}.
    \item \textbf{dni}: DNI del alumno \texttt{(String)}.
    \item \textbf{modulo\_ids}: Relación con módulos \texttt{(One2many)}.

\end{itemize}

\justify
El campo \texttt{modulo\_ids} es una vista de lista para mostrar los datos de los módulos que cursa el alumno. Se mostrarán los siguientes datos:

\begin{itemize}

    \item \textbf{name}: Nombre completo del módulo \texttt{(String)}.
    \item \textbf{codigo}: Código del módulo \texttt{(String)}.
    \item \textbf{ciclo\_ids}: Ciclo Formativo del módulo \texttt{(One2Many)}.
    \item \textbf{profesor\_ids}: Profesor del módulo \texttt{(One2Many)}.

\end{itemize}

\newpage

\justify
Así quedaría la vista \textbf{form} \texttt{(views/alumno\_views.xml)}:

\begin{figure}[h]

    \centering
    \includegraphics[width=13cm]{images/48.png}

\end{figure}

\newpage

\subsubsection{Vistas de Profesor}

\justify
El primer paso sera definir la acción principal que será abrir el modelo \texttt{ciclos.profesor} y muestra sus vistas. Se deben de definir los siguientes campos:

\begin{itemize}

    \item \textbf{name}: Título visible de la ventana.
    \item \textbf{res\_model}: Modelo al que se refiere la acción.
    \item \textbf{view\_mode}: Tipo de vistas que se mostrarán: list \texttt{(tabla)}, form \texttt{(formulario)}.

\end{itemize}

\begin{figure}[h]

    \centering
    \includegraphics[width=13cm]{images/40.png}

\end{figure}

\justify
Una vez definida la acción principal, se procederá a crear y explicar todas las vistas del modelo. La primera de ellas será la vista de tipo \textbf{list}, que mostrará los profesores en una tabla. Para ello se definen los siguientes elementos:

\begin{itemize}

    \item \textbf{name}: Título visible de la ventana.
    \item \textbf{model}: Modelo al que pertenece esta vista.
    \item \textbf{arch}: Estructura XML de la vista.

\end{itemize}

\justify
Dentro del campo con \texttt{name: arch} se define el tipo de vista que en este caso será de tipo \textbf{list} y también se definen los campos que se mostrarán en la vista:

\begin{itemize}

    \item \textbf{name}: Nombre completo del profesor \texttt{(String)}.
    \item \textbf{dni}: DNI del profesor \texttt{(String)}.

\end{itemize}

\newpage

\justify
Así quedaría la vista \textbf{list} \texttt{(views/profesor\_views.xml)}:

\begin{figure}[h]

    \centering
    \includegraphics[width=13cm]{images/44.png}

\end{figure}

\justify
La última vista a explicar de este modelo será la vista de tipo \textbf{form}, utilizada para crear y editar profesores. Se definen los siguientes elementos:

\begin{itemize}

    \item \textbf{name}: Título visible de la ventana.
    \item \textbf{model}: Modelo al que pertenece esta vista.
    \item \textbf{arch}: Estructura XML de la vista.

\end{itemize}

\justify
Dentro del campo con \texttt{name: arch} se define el tipo de vista que en este caso será de tipo \textbf{form}. Se utliza una etiqueta \textbf{sheet} como contenedor principal del formulario y también se definen los campos que se mostrarán en la vista para poder ser rellenados:

\begin{itemize}

    \item \textbf{name}: Nombre completo del profesor \texttt{(String)}.
    \item \textbf{dni}: DNI del profesor \texttt{(String)}.
    \item \textbf{modulo\_ids}: Relación con módulos \texttt{(One2many)}.

\end{itemize}

\justify
El campo \texttt{modulo\_ids} es una vista de lista para mostrar los datos de los módulos que imparte el profesor. Se mostrarán los siguientes datos:

\begin{itemize}

    \item \textbf{name}: Nombre completo del módulo \texttt{(String)}.
    \item \textbf{codigo}: Código del módulo \texttt{(String)}.
    \item \textbf{ciclo\_ids}: Ciclo Formativo del módulo \texttt{(One2Many)}.

\end{itemize}

\newpage

\justify
Así quedaría la vista \textbf{form} \texttt{(views/profesor\_views.xml)}:

\begin{figure}[h]

    \centering
    \includegraphics[width=13cm]{images/49.png}

\end{figure}

\newpage

\subsubsection{Vistas de Ciclo Formativo}

\justify
El primer paso sera definir la acción principal que será abrir el modelo \texttt{ciclos.ciclo} y muestra sus vistas. Se deben de definir los siguientes campos:

\begin{itemize}

    \item \textbf{name}: Título visible de la ventana.
    \item \textbf{res\_model}: Modelo al que se refiere la acción.
    \item \textbf{view\_mode}: Tipo de vistas que se mostrarán: list \texttt{(tabla)}, form \texttt{(formulario)}.

\end{itemize}

\begin{figure}[h]

    \centering
    \includegraphics[width=13cm]{images/41.png}

\end{figure}

\justify
Una vez definida la acción principal, se procederá a crear y explicar todas las vistas del modelo. La primera de ellas será la vista de tipo \textbf{list}, que mostrará las consultas en una tabla. Para ello se definen los siguientes elementos:

\begin{itemize}

    \item \textbf{name}: Título visible de la ventana.
    \item \textbf{model}: Modelo al que pertenece esta vista.
    \item \textbf{arch}: Estructura XML de la vista.

\end{itemize}

\justify
Dentro del campo con \texttt{name: arch} se define el tipo de vista que en este caso será de tipo \textbf{list} y también se definen los campos que se mostrarán en la vista:

\begin{itemize}

    \item \textbf{name}: Nombre completo del ciclo formativo \texttt{(String)}.
    \item \textbf{codigo}: Código identificador del ciclo formativo \texttt{(String)}.

\end{itemize}

\newpage

\justify
Así quedaría la vista \textbf{list} \texttt{(views/ciclo\_formativo\_views.xml)}:

\begin{figure}[h]

    \centering
    \includegraphics[width=13cm]{images/45.png}

\end{figure}

\justify
La última vista a explicar de este modelo será la vista de tipo \textbf{form}, utilizada para crear y editar los ciclos formativos. Se definen los siguientes elementos:

\begin{itemize}

    \item \textbf{name}: Título visible de la ventana.
    \item \textbf{model}: Modelo al que pertenece esta vista.
    \item \textbf{arch}: Estructura XML de la vista.

\end{itemize}

\justify
Dentro del campo con \texttt{name: arch} se define el tipo de vista que en este caso será de tipo \textbf{form}. Se utliza una etiqueta \textbf{sheet} como contenedor principal del formulario y también se definen los campos que se mostrarán en la vista para poder ser rellenados:

\begin{itemize}

    \item \textbf{name}: Nombre completo del ciclo \texttt{(String)}.
    \item \textbf{codigo}: Código identificador del ciclo \texttt{(String)}.
    \item \textbf{modulo\_ids}: Relación con módulos \texttt{(One2many)}.

\end{itemize}

\justify
El campo \texttt{modulo\_ids} es una vista de lista para mostrar los datos de los módulos que pertenecen al ciclo. Se mostrarán los siguientes datos:

\begin{itemize}

    \item \textbf{name}: Nombre completo del módulo \texttt{(String)}.
    \item \textbf{codigo}: Código del módulo \texttt{(String)}.
    \item \textbf{profesor\_id}: Profesor del módulo \texttt{(One2Many)}.
    \item \textbf{alumno\_ids}: Alumnos del módulo \texttt{(Many2Many)}.

\end{itemize}

\newpage

\justify
Así quedaría la vista \textbf{form} \texttt{(views/ciclo\_formativo\_views.xml)}:

\begin{figure}[h]

    \centering
    \includegraphics[width=13cm]{images/50.png}

\end{figure}

\newpage

\subsubsection{Vistas de Módulo}

\justify
El primer paso sera definir la acción principal que será abrir el modelo \texttt{ciclos.modulo} y muestra sus vistas. Se deben de definir los siguientes campos:

\begin{itemize}

    \item \textbf{name}: Título visible de la ventana.
    \item \textbf{res\_model}: Modelo al que se refiere la acción.
    \item \textbf{view\_mode}: Tipo de vistas que se mostrarán: list \texttt{(tabla)}, form \texttt{(formulario)}.

\end{itemize}

\begin{figure}[h]

    \centering
    \includegraphics[width=13cm]{images/42.png}

\end{figure}

\justify
Una vez definida la acción principal, se procederá a crear y explicar todas las vistas del modelo. La primera de ellas será la vista de tipo \textbf{list}, que mostrará las consultas en una tabla. Para ello se definen los siguientes elementos:

\begin{itemize}

    \item \textbf{name}: Título visible de la ventana.
    \item \textbf{model}: Modelo al que pertenece esta vista.
    \item \textbf{arch}: Estructura XML de la vista.

\end{itemize}

\justify
Dentro del campo con \texttt{name: arch} se define el tipo de vista que en este caso será de tipo \textbf{list} y también se definen los campos que se mostrarán en la vista:

\begin{itemize}

    \item \textbf{name}: Nombre completo del módulo \texttt{(String)}.
    \item \textbf{codigo}: Código identificador del módulo \texttt{(String)}.
    \item \textbf{ciclo\_id}: Ciclo formativo al que pertenece el módulo \texttt{(Many2one)}.
    \item \textbf{profesor\_id}: Profesor que imparte el módulo \texttt{(Many2one)}.
    \item \textbf{alumno\_ids}: Alumnos que cursan el módulo \texttt{(Many2many)}.

\end{itemize}

\newpage

\justify
Así quedaría la vista \textbf{list} \texttt{(views/modulo\_views.xml)}:

\begin{figure}[h]

    \centering
    \includegraphics[width=13cm]{images/46.png}

\end{figure}

\justify
La última vista a explicar de este modelo será la vista de tipo \textbf{form}, utilizada para crear y editar módulos pertenecientes a un ciclo formativo. Se definen los siguientes elementos:

\begin{itemize}

    \item \textbf{name}: Título visible de la ventana.
    \item \textbf{model}: Modelo al que pertenece esta vista.
    \item \textbf{arch}: Estructura XML de la vista.

\end{itemize}

\justify
Dentro del campo con \texttt{name: arch} se define el tipo de vista que en este caso será de tipo \textbf{form}. Se utliza una etiqueta \textbf{sheet} como contenedor principal del formulario y también se definen los campos que se mostrarán en la vista para poder ser rellenados:

\begin{itemize}

    \item \textbf{name}: Nombre completo del módulo \texttt{(String)}.
    \item \textbf{codigo}: Código identificador del módulo \texttt{(String)}.
    \item \textbf{ciclo\_id}: Ciclo formativo al que pertenece el módulo \texttt{(Many2one)}.
    \item \textbf{profesor\_id}: Profesor que imparte el módulo \texttt{(Many2one)}.
    \item \textbf{alumno\_ids}: Alumnos que cursan el módulo \texttt{(Many2many)}.

\end{itemize}

\justify
El campo \texttt{alumno\_ids} es una vista de lista para mostrar los datos de los alumnos que cursan el módulo. Se mostrarán los siguientes datos:

\begin{itemize}

    \item \textbf{name}: Nombre completo del alumno \texttt{(String)}.
    \item \textbf{dni}: DNI del alumno \texttt{(String)}.

\end{itemize}

\newpage

\justify
Así quedaría la vista \textbf{form} \texttt{(views/modulo\_views.xml)}:

\begin{figure}[h]

    \centering
    \includegraphics[width=13cm]{images/50.png}

\end{figure}

\newpage

\subsubsection{Vistas de Menús}

\justify
A continuación se definirán los menús del módulo que aparecerán en la parte superior de la interfaz de Odoo y para ello se les debe asignar un atributo \textbf{id} que será el identificador único de cada menú, un atributo \textbf{name} que será el nombre visble del menú y un atributo \textbf{parent} que será el menú padre del menú. 

\justify
Para ello, se definirán los siguientes \texttt{menuitem}:

\begin{itemize}

    \item \textbf{menu\_ciclos\_principal}: Menú principal que aparece en la barra de navegación.
    \item \textbf{menu\_alumnos}: Submenú del modelo Alumno que aparece en el menú principal.
    \item \textbf{menu\_profesores}: Submenú del modelo médico que aparece en el menú principal.
    \item \textbf{menu\_ciclos}: Submenú del modelo consulta que aparece en el menú principal.

\end{itemize}

\justify
Así quedarían los menús \texttt{(views/menu\_views.xml)}:

\begin{figure}[h]

    \centering
    \includegraphics[width=15cm]{images/47.png}

\end{figure}

\newpage

\subsection{Otros archivos}

\justify
Se debe modificar el archivo \texttt{models/\_\_init\_\_.py} para importar los modelos que se han creado.

\begin{figure}[h]

    \centering
    \includegraphics[width=7cm]{images/51.png}

\end{figure}

\justify
Se debe modificar el archivo \texttt{\_\_manifest\_\_.py} para importar las vistas que se han creado.

\begin{figure}[h]

    \centering
    \includegraphics[width=10cm]{images/52.png}

\end{figure}

\subsection{Funcionamiento}

\justify
Ya terminada la explicación de los modelos y las vistas del módulo, se pondrá a prueba su funcionamiento y para ello el primer paso es pasar el directorio del módulo del equipo local al contenedor de Odoo, concretamente a la carpeta mapeada para almacenar los módulos extra añadidos por el usuario y para ello se utilizará el siguiente comando: \newline

\centering
\texttt{docker cp <ruta\_módulo> <contenedor\_odoo>:/mnt/extra-addons/<directorio>}

\begin{figure}[h]

    \centering
    \includegraphics[width=15cm]{images/61.png}

\end{figure}

\newpage

\justify
Una vez que se haya copiado el módulo al contenedor de Odoo, se debe activar el \textbf{Modo Desarrollador} y actualizar la lista de aplicaciones (explicado en la práctica anterior). Posteriormente se debe buscar el módulo en la lista de aplicaciones e instalarlo pulsando sobre el botón lila de \textbf{Activar}.

\begin{figure}[h]

    \centering
    \includegraphics[width=15cm]{images/62.png}

\end{figure}

\justify
Una vez instalado el módulo, se accederá a él desde la barra de navegación de Odoo y se empezará por crear un nuevo ciclo formativo. La vista de creación y edición de ciclos formativos \textbf{(form)}, se ve de la siguiente manera (aunque se puedan crear los módulos del ciclo desde esta vista, de momento se creará el ciclo sin ningún módulo):

\begin{figure}[h]

    \centering
    \includegraphics[width=15cm]{images/63.png}

\end{figure}

\justify
Una vez creado el ciclo formativo, se puede crearán nuevos profesores para impartir módulos en este ciclo. La vista de creación y edición de profesores \textbf{(form)}, se ve de la siguiente manera (aunque se puedan crear los módulos que imparten los profesores desde esta vista, de momento se creará el profesor sin ningún módulo asignado):

\begin{figure}[h]

    \centering
    \includegraphics[width=15cm]{images/64.png}

\end{figure}

\newpage

\justify
Ya creados los docentes que impartirán los ciclos, se crearán los alumnos que asistirán a los módulos impartidos por los profesores. La vista de creación y edición de alumnos \textbf{(form)}, se ve de la siguiente manera (aunque se puedan crear los módulos a los que asisten los alumnos desde esta vista, de momento se creará el alumno sin ningún módulo asignado):

\begin{figure}[h]

    \centering
    \includegraphics[width=15cm]{images/65.png}

\end{figure}

\justify
Ahora se crearán los módulos que se impartirán en el ciclo formativo. La vista de creación y edición de módulos \textbf{(form)}, se ve de la siguiente manera:

\begin{figure}[h]

    \centering
    \includegraphics[width=15cm]{images/66.png}

\end{figure}

\newpage

\justify
Una vez creados todos los datos, así se verá la lista de módulos en el ciclo formativo:

\begin{figure}[h]

    \centering
    \includegraphics[width=15cm]{images/67.png}

\end{figure}

\justify
Así se ve la vista de lista de los profesores:

\begin{figure}[h]

    \centering
    \includegraphics[width=10cm]{images/68.png}

\end{figure}

\newpage

\justify
Así se verá la lista de los alumnos:

\begin{figure}[h]

    \centering
    \includegraphics[width=10cm]{images/69.png}

\end{figure}

\justify
Y por último, así se verá la lista de ciclos formativos:

\begin{figure}[h]

    \centering
    \includegraphics[width=10cm]{images/70.png}

\end{figure}


\end{document}